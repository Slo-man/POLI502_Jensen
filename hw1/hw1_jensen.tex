% Options for packages loaded elsewhere
\PassOptionsToPackage{unicode}{hyperref}
\PassOptionsToPackage{hyphens}{url}
\PassOptionsToPackage{dvipsnames,svgnames,x11names}{xcolor}
%
\documentclass[
  letterpaper,
  DIV=11,
  numbers=noendperiod]{scrartcl}

\usepackage{amsmath,amssymb}
\usepackage{iftex}
\ifPDFTeX
  \usepackage[T1]{fontenc}
  \usepackage[utf8]{inputenc}
  \usepackage{textcomp} % provide euro and other symbols
\else % if luatex or xetex
  \usepackage{unicode-math}
  \defaultfontfeatures{Scale=MatchLowercase}
  \defaultfontfeatures[\rmfamily]{Ligatures=TeX,Scale=1}
\fi
\usepackage{lmodern}
\ifPDFTeX\else  
    % xetex/luatex font selection
\fi
% Use upquote if available, for straight quotes in verbatim environments
\IfFileExists{upquote.sty}{\usepackage{upquote}}{}
\IfFileExists{microtype.sty}{% use microtype if available
  \usepackage[]{microtype}
  \UseMicrotypeSet[protrusion]{basicmath} % disable protrusion for tt fonts
}{}
\makeatletter
\@ifundefined{KOMAClassName}{% if non-KOMA class
  \IfFileExists{parskip.sty}{%
    \usepackage{parskip}
  }{% else
    \setlength{\parindent}{0pt}
    \setlength{\parskip}{6pt plus 2pt minus 1pt}}
}{% if KOMA class
  \KOMAoptions{parskip=half}}
\makeatother
\usepackage{xcolor}
\setlength{\emergencystretch}{3em} % prevent overfull lines
\setcounter{secnumdepth}{-\maxdimen} % remove section numbering
% Make \paragraph and \subparagraph free-standing
\makeatletter
\ifx\paragraph\undefined\else
  \let\oldparagraph\paragraph
  \renewcommand{\paragraph}{
    \@ifstar
      \xxxParagraphStar
      \xxxParagraphNoStar
  }
  \newcommand{\xxxParagraphStar}[1]{\oldparagraph*{#1}\mbox{}}
  \newcommand{\xxxParagraphNoStar}[1]{\oldparagraph{#1}\mbox{}}
\fi
\ifx\subparagraph\undefined\else
  \let\oldsubparagraph\subparagraph
  \renewcommand{\subparagraph}{
    \@ifstar
      \xxxSubParagraphStar
      \xxxSubParagraphNoStar
  }
  \newcommand{\xxxSubParagraphStar}[1]{\oldsubparagraph*{#1}\mbox{}}
  \newcommand{\xxxSubParagraphNoStar}[1]{\oldsubparagraph{#1}\mbox{}}
\fi
\makeatother

\usepackage{color}
\usepackage{fancyvrb}
\newcommand{\VerbBar}{|}
\newcommand{\VERB}{\Verb[commandchars=\\\{\}]}
\DefineVerbatimEnvironment{Highlighting}{Verbatim}{commandchars=\\\{\}}
% Add ',fontsize=\small' for more characters per line
\usepackage{framed}
\definecolor{shadecolor}{RGB}{241,243,245}
\newenvironment{Shaded}{\begin{snugshade}}{\end{snugshade}}
\newcommand{\AlertTok}[1]{\textcolor[rgb]{0.68,0.00,0.00}{#1}}
\newcommand{\AnnotationTok}[1]{\textcolor[rgb]{0.37,0.37,0.37}{#1}}
\newcommand{\AttributeTok}[1]{\textcolor[rgb]{0.40,0.45,0.13}{#1}}
\newcommand{\BaseNTok}[1]{\textcolor[rgb]{0.68,0.00,0.00}{#1}}
\newcommand{\BuiltInTok}[1]{\textcolor[rgb]{0.00,0.23,0.31}{#1}}
\newcommand{\CharTok}[1]{\textcolor[rgb]{0.13,0.47,0.30}{#1}}
\newcommand{\CommentTok}[1]{\textcolor[rgb]{0.37,0.37,0.37}{#1}}
\newcommand{\CommentVarTok}[1]{\textcolor[rgb]{0.37,0.37,0.37}{\textit{#1}}}
\newcommand{\ConstantTok}[1]{\textcolor[rgb]{0.56,0.35,0.01}{#1}}
\newcommand{\ControlFlowTok}[1]{\textcolor[rgb]{0.00,0.23,0.31}{\textbf{#1}}}
\newcommand{\DataTypeTok}[1]{\textcolor[rgb]{0.68,0.00,0.00}{#1}}
\newcommand{\DecValTok}[1]{\textcolor[rgb]{0.68,0.00,0.00}{#1}}
\newcommand{\DocumentationTok}[1]{\textcolor[rgb]{0.37,0.37,0.37}{\textit{#1}}}
\newcommand{\ErrorTok}[1]{\textcolor[rgb]{0.68,0.00,0.00}{#1}}
\newcommand{\ExtensionTok}[1]{\textcolor[rgb]{0.00,0.23,0.31}{#1}}
\newcommand{\FloatTok}[1]{\textcolor[rgb]{0.68,0.00,0.00}{#1}}
\newcommand{\FunctionTok}[1]{\textcolor[rgb]{0.28,0.35,0.67}{#1}}
\newcommand{\ImportTok}[1]{\textcolor[rgb]{0.00,0.46,0.62}{#1}}
\newcommand{\InformationTok}[1]{\textcolor[rgb]{0.37,0.37,0.37}{#1}}
\newcommand{\KeywordTok}[1]{\textcolor[rgb]{0.00,0.23,0.31}{\textbf{#1}}}
\newcommand{\NormalTok}[1]{\textcolor[rgb]{0.00,0.23,0.31}{#1}}
\newcommand{\OperatorTok}[1]{\textcolor[rgb]{0.37,0.37,0.37}{#1}}
\newcommand{\OtherTok}[1]{\textcolor[rgb]{0.00,0.23,0.31}{#1}}
\newcommand{\PreprocessorTok}[1]{\textcolor[rgb]{0.68,0.00,0.00}{#1}}
\newcommand{\RegionMarkerTok}[1]{\textcolor[rgb]{0.00,0.23,0.31}{#1}}
\newcommand{\SpecialCharTok}[1]{\textcolor[rgb]{0.37,0.37,0.37}{#1}}
\newcommand{\SpecialStringTok}[1]{\textcolor[rgb]{0.13,0.47,0.30}{#1}}
\newcommand{\StringTok}[1]{\textcolor[rgb]{0.13,0.47,0.30}{#1}}
\newcommand{\VariableTok}[1]{\textcolor[rgb]{0.07,0.07,0.07}{#1}}
\newcommand{\VerbatimStringTok}[1]{\textcolor[rgb]{0.13,0.47,0.30}{#1}}
\newcommand{\WarningTok}[1]{\textcolor[rgb]{0.37,0.37,0.37}{\textit{#1}}}

\providecommand{\tightlist}{%
  \setlength{\itemsep}{0pt}\setlength{\parskip}{0pt}}\usepackage{longtable,booktabs,array}
\usepackage{calc} % for calculating minipage widths
% Correct order of tables after \paragraph or \subparagraph
\usepackage{etoolbox}
\makeatletter
\patchcmd\longtable{\par}{\if@noskipsec\mbox{}\fi\par}{}{}
\makeatother
% Allow footnotes in longtable head/foot
\IfFileExists{footnotehyper.sty}{\usepackage{footnotehyper}}{\usepackage{footnote}}
\makesavenoteenv{longtable}
\usepackage{graphicx}
\makeatletter
\newsavebox\pandoc@box
\newcommand*\pandocbounded[1]{% scales image to fit in text height/width
  \sbox\pandoc@box{#1}%
  \Gscale@div\@tempa{\textheight}{\dimexpr\ht\pandoc@box+\dp\pandoc@box\relax}%
  \Gscale@div\@tempb{\linewidth}{\wd\pandoc@box}%
  \ifdim\@tempb\p@<\@tempa\p@\let\@tempa\@tempb\fi% select the smaller of both
  \ifdim\@tempa\p@<\p@\scalebox{\@tempa}{\usebox\pandoc@box}%
  \else\usebox{\pandoc@box}%
  \fi%
}
% Set default figure placement to htbp
\def\fps@figure{htbp}
\makeatother

\KOMAoption{captions}{tableheading}
\makeatletter
\@ifpackageloaded{caption}{}{\usepackage{caption}}
\AtBeginDocument{%
\ifdefined\contentsname
  \renewcommand*\contentsname{Table of contents}
\else
  \newcommand\contentsname{Table of contents}
\fi
\ifdefined\listfigurename
  \renewcommand*\listfigurename{List of Figures}
\else
  \newcommand\listfigurename{List of Figures}
\fi
\ifdefined\listtablename
  \renewcommand*\listtablename{List of Tables}
\else
  \newcommand\listtablename{List of Tables}
\fi
\ifdefined\figurename
  \renewcommand*\figurename{Figure}
\else
  \newcommand\figurename{Figure}
\fi
\ifdefined\tablename
  \renewcommand*\tablename{Table}
\else
  \newcommand\tablename{Table}
\fi
}
\@ifpackageloaded{float}{}{\usepackage{float}}
\floatstyle{ruled}
\@ifundefined{c@chapter}{\newfloat{codelisting}{h}{lop}}{\newfloat{codelisting}{h}{lop}[chapter]}
\floatname{codelisting}{Listing}
\newcommand*\listoflistings{\listof{codelisting}{List of Listings}}
\makeatother
\makeatletter
\makeatother
\makeatletter
\@ifpackageloaded{caption}{}{\usepackage{caption}}
\@ifpackageloaded{subcaption}{}{\usepackage{subcaption}}
\makeatother

\usepackage{bookmark}

\IfFileExists{xurl.sty}{\usepackage{xurl}}{} % add URL line breaks if available
\urlstyle{same} % disable monospaced font for URLs
\hypersetup{
  pdftitle={Aric\_Jensen\_HW1},
  colorlinks=true,
  linkcolor={blue},
  filecolor={Maroon},
  citecolor={Blue},
  urlcolor={Blue},
  pdfcreator={LaTeX via pandoc}}


\title{Aric\_Jensen\_HW1}
\author{}
\date{}

\begin{document}
\maketitle


\begin{Shaded}
\begin{Highlighting}[]
\DocumentationTok{\#\#{-}{-}{-}{-}{-}{-}{-}{-}{-}{-}{-}{-}{-}{-}{-}{-}{-}{-}{-}{-}{-}{-}{-}{-}{-}{-}{-}{-}{-}{-}{-}{-}{-}{-}{-}{-}{-}{-}{-}{-}{-}{-}{-}{-}{-}{-}{-}{-}{-}{-}{-}{-}{-}{-}{-}{-}{-}{-}{-}{-}{-}{-}{-}{-}{-}{-}{-}{-}{-}{-}{-}{-}{-}}
\DocumentationTok{\#\# R code for 502 Lab, week 4: Individual Exercises}
\DocumentationTok{\#\# Howard Liu}
\DocumentationTok{\#\# University of South Carolina}
\DocumentationTok{\#\#{-}{-}{-}{-}{-}{-}{-}{-}{-}{-}{-}{-}{-}{-}{-}{-}{-}{-}{-}{-}{-}{-}{-}{-}{-}{-}{-}{-}{-}{-}{-}{-}{-}{-}{-}{-}{-}{-}{-}{-}{-}{-}{-}{-}{-}{-}{-}{-}{-}{-}{-}{-}{-}{-}{-}{-}{-}{-}{-}{-}{-}{-}{-}{-}{-}{-}{-}{-}{-}{-}{-}{-}{-}}




\CommentTok{\# 1. Functions {-}{-}{-}{-}{-}{-}{-}{-}{-}{-}{-}{-}{-}{-}{-}{-}{-}{-}{-}{-}{-}{-}{-}{-}{-}{-}{-}{-}{-}{-}{-}{-}{-}{-}{-}{-}{-}{-}{-}{-}{-}{-}{-}{-}{-}{-}{-}{-}{-}{-}{-}{-}{-}{-}{-}{-}{-}{-}{-}{-}}


\CommentTok{\# What is a function that displays all the objects currently stored in }
\CommentTok{\# the memory? Write it below after deleting the line that reads }
\CommentTok{\# "WRITE YOUR ANSWER (code) HERE", and execute it. }


\FunctionTok{ls}\NormalTok{()}
\end{Highlighting}
\end{Shaded}

\begin{verbatim}
character(0)
\end{verbatim}

\begin{Shaded}
\begin{Highlighting}[]
\CommentTok{\# Create a new object named x5 that is a number 100. }


\NormalTok{x5 }\OtherTok{=} \DecValTok{100}

\CommentTok{\# Calculate the square root of x5 using the sqrt() function}


\FunctionTok{sqrt}\NormalTok{(x5)}
\end{Highlighting}
\end{Shaded}

\begin{verbatim}
[1] 10
\end{verbatim}

\begin{Shaded}
\begin{Highlighting}[]
\CommentTok{\# sqrt(x5)}
\CommentTok{\# [1] 10}

\CommentTok{\# Calculate the square root of x5 by raising it to the power of 0.5. }
\CommentTok{\# Your numeric answer should be exactly the same as when you used the }
\CommentTok{\# sqrt() function. This is because taking the sqaure root of something }
\CommentTok{\# is equivalent to raising it to the power of 0.5. }

\NormalTok{(x5)}\SpecialCharTok{\^{}}\NormalTok{.}\DecValTok{5}
\end{Highlighting}
\end{Shaded}

\begin{verbatim}
[1] 10
\end{verbatim}

\begin{Shaded}
\begin{Highlighting}[]
\CommentTok{\# (x5)\^{}.5}
\CommentTok{\# [1] 10}


\CommentTok{\# Create an object called x6 that is equal to 31.8734.}


\NormalTok{x6}\OtherTok{=}\FloatTok{31.8734}


\CommentTok{\# Use the round() function to get the value of x6 rounded off to }
\CommentTok{\# three decimal places}


\FunctionTok{round}\NormalTok{(x6,}\AttributeTok{digits=}\DecValTok{3}\NormalTok{)}
\end{Highlighting}
\end{Shaded}

\begin{verbatim}
[1] 31.873
\end{verbatim}

\begin{Shaded}
\begin{Highlighting}[]
\CommentTok{\# round(x6,digits=3)}
\CommentTok{\# [1] 31.873}

\CommentTok{\# Functions floor() and ceiling() can also be used to trim a number}
\CommentTok{\# down to an integer: apply both of these functions to x6 and compare}
\CommentTok{\# the outputs. Can you guess what these functions do? }


\FunctionTok{floor}\NormalTok{(x6)}
\end{Highlighting}
\end{Shaded}

\begin{verbatim}
[1] 31
\end{verbatim}

\begin{Shaded}
\begin{Highlighting}[]
\CommentTok{\# floor(x6)}
\CommentTok{\# [1] 31}

\FunctionTok{ceiling}\NormalTok{(x6)}
\end{Highlighting}
\end{Shaded}

\begin{verbatim}
[1] 32
\end{verbatim}

\begin{Shaded}
\begin{Highlighting}[]
\CommentTok{\# ceiling(x6)}
\CommentTok{\# [1] 32}

\CommentTok{\# floor is round down and ceiling is round up}


\CommentTok{\# To find out if your hunch was right, refer to the help file of these}
\CommentTok{\# functions. Write a code to open up the help file for the floor function.}


\FunctionTok{help}\NormalTok{(floor)}
\end{Highlighting}
\end{Shaded}

\begin{verbatim}
starting httpd help server ... done
\end{verbatim}

\begin{Shaded}
\begin{Highlighting}[]
\CommentTok{\# 2. Vectors {-}{-}{-}{-}{-}{-}{-}{-}{-}{-}{-}{-}{-}{-}{-}{-}{-}{-}{-}{-}{-}{-}{-}{-}{-}{-}{-}{-}{-}{-}{-}{-}{-}{-}{-}{-}{-}{-}{-}{-}{-}{-}{-}{-}{-}{-}{-}{-}{-}{-}{-}{-}{-}{-}{-}{-}{-}{-}{-}{-}{-}{-}}


\CommentTok{\# Create an object called "vec.a" which is a vector consisting of }
\CommentTok{\# the numbers, 1, 3, 5, 7. You need to use the c function. }


\NormalTok{vec.a}\OtherTok{=}\FunctionTok{c}\NormalTok{(}\DecValTok{1}\NormalTok{,}\DecValTok{3}\NormalTok{,}\DecValTok{5}\NormalTok{,}\DecValTok{7}\NormalTok{)}


\CommentTok{\# Create a vector called "vec.b" consisting of the numbers, 2, 4, 6, 8.}


\NormalTok{vec.b}\OtherTok{=}\FunctionTok{c}\NormalTok{(}\DecValTok{2}\NormalTok{,}\DecValTok{4}\NormalTok{,}\DecValTok{6}\NormalTok{,}\DecValTok{8}\NormalTok{)}


\CommentTok{\# Subtract vec.b from vec.a}


\NormalTok{vec.b}\SpecialCharTok{{-}}\NormalTok{vec.a}
\end{Highlighting}
\end{Shaded}

\begin{verbatim}
[1] 1 1 1 1
\end{verbatim}

\begin{Shaded}
\begin{Highlighting}[]
\CommentTok{\# vec.b{-}vec.a}
\CommentTok{\# [1] 1 1 1 1}

\CommentTok{\# Create a new vector called vec.c by multiplying vec.a by vector vec.b}


\NormalTok{vec.c}\OtherTok{=}\NormalTok{vec.a}\SpecialCharTok{*}\NormalTok{vec.b}
\FunctionTok{print}\NormalTok{(vec.c)}
\end{Highlighting}
\end{Shaded}

\begin{verbatim}
[1]  2 12 30 56
\end{verbatim}

\begin{Shaded}
\begin{Highlighting}[]
\CommentTok{\# [1]  2 12 30 56}

\CommentTok{\# Create a new vector called vec.d by taking the square root of each }
\CommentTok{\# member of vec.c}


\NormalTok{vec.d}\OtherTok{=}\FunctionTok{sqrt}\NormalTok{(vec.c)}
\FunctionTok{print}\NormalTok{(vec.d)}
\end{Highlighting}
\end{Shaded}

\begin{verbatim}
[1] 1.414214 3.464102 5.477226 7.483315
\end{verbatim}

\begin{Shaded}
\begin{Highlighting}[]
\CommentTok{\# [1] 1.414214 3.464102 5.477226 7.483315}

\CommentTok{\# What is the third element of the vec.d vector? Find out using }
\CommentTok{\# square bracket. Note that since this is a vector, you only need to }
\CommentTok{\# provide a single number inside the brackets. }


\NormalTok{vec.d[}\DecValTok{3}\NormalTok{]}
\end{Highlighting}
\end{Shaded}

\begin{verbatim}
[1] 5.477226
\end{verbatim}

\begin{Shaded}
\begin{Highlighting}[]
\CommentTok{\# vec.d[3]}
\CommentTok{\# [1] 5.477226}



\CommentTok{\# Create a new vector called vec.e consisting of all the integers }
\CommentTok{\# from 1 through 100. You should use the seq function, rather than writing }
\CommentTok{\# down all the 100 integers individually. }


\NormalTok{vec.e}\OtherTok{=}\FunctionTok{seq}\NormalTok{(}\AttributeTok{from=}\DecValTok{1}\NormalTok{,}\AttributeTok{to=}\DecValTok{100}\NormalTok{)}
\FunctionTok{print}\NormalTok{(vec.e)}
\end{Highlighting}
\end{Shaded}

\begin{verbatim}
  [1]   1   2   3   4   5   6   7   8   9  10  11  12  13  14  15  16  17  18
 [19]  19  20  21  22  23  24  25  26  27  28  29  30  31  32  33  34  35  36
 [37]  37  38  39  40  41  42  43  44  45  46  47  48  49  50  51  52  53  54
 [55]  55  56  57  58  59  60  61  62  63  64  65  66  67  68  69  70  71  72
 [73]  73  74  75  76  77  78  79  80  81  82  83  84  85  86  87  88  89  90
 [91]  91  92  93  94  95  96  97  98  99 100
\end{verbatim}

\begin{Shaded}
\begin{Highlighting}[]
\FunctionTok{print}\NormalTok{(vec.e)}
\end{Highlighting}
\end{Shaded}

\begin{verbatim}
  [1]   1   2   3   4   5   6   7   8   9  10  11  12  13  14  15  16  17  18
 [19]  19  20  21  22  23  24  25  26  27  28  29  30  31  32  33  34  35  36
 [37]  37  38  39  40  41  42  43  44  45  46  47  48  49  50  51  52  53  54
 [55]  55  56  57  58  59  60  61  62  63  64  65  66  67  68  69  70  71  72
 [73]  73  74  75  76  77  78  79  80  81  82  83  84  85  86  87  88  89  90
 [91]  91  92  93  94  95  96  97  98  99 100
\end{verbatim}

\begin{Shaded}
\begin{Highlighting}[]
\CommentTok{\# [1]   1   2   3   4   5   6   7   8   9  10  11  12  13  14  15  16  17  18  19  20}
\CommentTok{\# [21]  21  22  23  24  25  26  27  28  29  30  31  32  33  34  35  36  37  38  39  40}
\CommentTok{\# [41]  41  42  43  44  45  46  47  48  49  50  51  52  53  54  55  56  57  58  59  60}
\CommentTok{\# [61]  61  62  63  64  65  66  67  68  69  70  71  72  73  74  75  76  77  78  79  80}
\CommentTok{\# [81]  81  82  83  84  85  86  87  88  89  90  91  92  93  94  95  96  97  98  99 100}

\CommentTok{\# The mean function calculates the arithmetic mean of the numbers stored }
\CommentTok{\# in an object. Using the mean function, calculate the mean of the vec.e vector.}


\FunctionTok{mean}\NormalTok{(vec.e)}
\end{Highlighting}
\end{Shaded}

\begin{verbatim}
[1] 50.5
\end{verbatim}

\begin{Shaded}
\begin{Highlighting}[]
\CommentTok{\# [1] 50.5}

\CommentTok{\# As we saw in the joint exercise, the sum function calculates the sum of all }
\CommentTok{\# the elements in an object. Calculate the sum of the vec.e vector. }


\FunctionTok{sum}\NormalTok{(vec.e)}
\end{Highlighting}
\end{Shaded}

\begin{verbatim}
[1] 5050
\end{verbatim}

\begin{Shaded}
\begin{Highlighting}[]
\CommentTok{\# [1] 5050}

\CommentTok{\# The length function returns the number of elements stored in an object. }
\CommentTok{\# Using the length function, find the number of elements stored in the vec.e}
\CommentTok{\# vector. }


\FunctionTok{length}\NormalTok{(vec.e)}
\end{Highlighting}
\end{Shaded}

\begin{verbatim}
[1] 100
\end{verbatim}

\begin{Shaded}
\begin{Highlighting}[]
\CommentTok{\# [1] 100}

\CommentTok{\# The mean of an object can be obtained by sum(X)/length(X) because}
\CommentTok{\# the defininition of the mean is the sum of elements divided by the number of }
\CommentTok{\# elements. Now, using the sum and length functions, calculate the mean of }
\CommentTok{\# the vec.e vector. Compare the answer with that obtained with the mean function}


\FunctionTok{sum}\NormalTok{(vec.e)}\SpecialCharTok{/}\FunctionTok{length}\NormalTok{(vec.e)}
\end{Highlighting}
\end{Shaded}

\begin{verbatim}
[1] 50.5
\end{verbatim}

\begin{Shaded}
\begin{Highlighting}[]
\FunctionTok{sum}\NormalTok{(vec.e)}\SpecialCharTok{/}\FunctionTok{length}\NormalTok{(vec.e)}
\end{Highlighting}
\end{Shaded}

\begin{verbatim}
[1] 50.5
\end{verbatim}

\begin{Shaded}
\begin{Highlighting}[]
\CommentTok{\# [1] 50.5}



\CommentTok{\# We have learned that the by argument specifies an increment. For example, }


\FunctionTok{seq}\NormalTok{(}\AttributeTok{from =} \DecValTok{0}\NormalTok{, }\AttributeTok{to =} \DecValTok{10}\NormalTok{, }\AttributeTok{by =} \DecValTok{2}\NormalTok{)}
\end{Highlighting}
\end{Shaded}

\begin{verbatim}
[1]  0  2  4  6  8 10
\end{verbatim}

\begin{Shaded}
\begin{Highlighting}[]
\CommentTok{\# This creates a sequence that starts from 0 and ends with 10, and with }
\CommentTok{\# an increment of 2. }




\CommentTok{\# Now, create a new object called olympic which is a sequence that }
\CommentTok{\# starts from 1896 and ends with 2012, with an increment of 4. }


\NormalTok{olympic}\OtherTok{=}\FunctionTok{seq}\NormalTok{(}\AttributeTok{from=}\DecValTok{1896}\NormalTok{,}\AttributeTok{to=}\DecValTok{2012}\NormalTok{,}\AttributeTok{by=}\DecValTok{4}\NormalTok{)}


\CommentTok{\# How many elements does the olympic vector contain? That is, what is}
\CommentTok{\# the length of this vector? Find out by applying a function (not by}
\CommentTok{\# manually counting the number of elements). }


\FunctionTok{length}\NormalTok{(olympic)}
\end{Highlighting}
\end{Shaded}

\begin{verbatim}
[1] 30
\end{verbatim}

\begin{Shaded}
\begin{Highlighting}[]
\CommentTok{\# [1] 30}



\CommentTok{\# So there are 30 elements in the olympic vector. Display all the }
\CommentTok{\# elements contained in the olympic vector. These are the years}
\CommentTok{\# where olympic games were (supposed to be) held. Display the }
\CommentTok{\# contents of the olympic vector. }


\NormalTok{olympic}
\end{Highlighting}
\end{Shaded}

\begin{verbatim}
 [1] 1896 1900 1904 1908 1912 1916 1920 1924 1928 1932 1936 1940 1944 1948 1952
[16] 1956 1960 1964 1968 1972 1976 1980 1984 1988 1992 1996 2000 2004 2008 2012
\end{verbatim}

\begin{Shaded}
\begin{Highlighting}[]
\CommentTok{\# [1] 1896 1900 1904 1908 1912 1916 1920 1924 1928 1932 1936 1940 1944 1948 1952 1956}
\CommentTok{\# [17] 1960 1964 1968 1972 1976 1980 1984 1988 1992 1996 2000 2004 2008 2012}

\CommentTok{\# Find out how many olympic games will have been held by the year}
\CommentTok{\# 2400. Use the length and seq functions. }


\NormalTok{olympic2400}\OtherTok{=}\FunctionTok{seq}\NormalTok{(}\AttributeTok{from=}\DecValTok{1896}\NormalTok{,}\AttributeTok{to=}\DecValTok{2400}\NormalTok{,}\AttributeTok{by=}\DecValTok{4}\NormalTok{)}
\FunctionTok{length}\NormalTok{(olympic2400)}
\end{Highlighting}
\end{Shaded}

\begin{verbatim}
[1] 127
\end{verbatim}

\begin{Shaded}
\begin{Highlighting}[]
\CommentTok{\# [1] 127}


\CommentTok{\# 3. Matrices {-}{-}{-}{-}{-}{-}{-}{-}{-}{-}{-}{-}{-}{-}{-}{-}{-}{-}{-}{-}{-}{-}{-}{-}{-}{-}{-}{-}{-}{-}{-}{-}{-}{-}{-}{-}{-}{-}{-}{-}{-}{-}{-}{-}{-}{-}{-}{-}{-}{-}{-}{-}{-}{-}{-}{-}{-}{-}{-}{-}{-}}


\CommentTok{\# Create a new vector called "v1" consisting of the following numbers:}
\CommentTok{\# 1, 3, 5, 7, 9, 11}


\NormalTok{v1}\OtherTok{=}\FunctionTok{seq}\NormalTok{(}\AttributeTok{from=}\DecValTok{1}\NormalTok{,}\AttributeTok{to=}\DecValTok{11}\NormalTok{,}\AttributeTok{by=}\DecValTok{2}\NormalTok{)}


\CommentTok{\# Find out the length of this vector (Don\textquotesingle{}t count the numbers by hand; }
\CommentTok{\# use an appropriate function). }


\FunctionTok{length}\NormalTok{(v1)}
\end{Highlighting}
\end{Shaded}

\begin{verbatim}
[1] 6
\end{verbatim}

\begin{Shaded}
\begin{Highlighting}[]
\CommentTok{\# [1] 6}



\CommentTok{\# We will conver this vector into a matrix. That is, we will rearrange this}
\CommentTok{\# vector so that it will have two dimensions (rows and columns). }
\CommentTok{\# Since this vector has 6 numbers, if we want the matrix to have two }
\CommentTok{\# rows, how many columns will there be?}


\CommentTok{\# there will be 3 columns}




\CommentTok{\# Create a matrix called mat.v using the following command: }
\CommentTok{\# matrix(data = v1, nrow = 2)}


\NormalTok{mat.v}\OtherTok{=}\FunctionTok{matrix}\NormalTok{(}\AttributeTok{data=}\NormalTok{v1,}\AttributeTok{nrow=}\DecValTok{2}\NormalTok{)}


\CommentTok{\# Take a look at the contens of this matrix. }
\CommentTok{\# How many columns are there? }

\CommentTok{\# there are 3 columns}

\FunctionTok{print}\NormalTok{(mat.v)}
\end{Highlighting}
\end{Shaded}

\begin{verbatim}
     [,1] [,2] [,3]
[1,]    1    5    9
[2,]    3    7   11
\end{verbatim}

\begin{Shaded}
\begin{Highlighting}[]
\CommentTok{\#     [,1] [,2] [,3]}
\CommentTok{\#[1,]    1    5    9}
\CommentTok{\#[2,]    3    7   11}

\CommentTok{\# Notice how the numbers in vec.v are used to fill up the cells of mat.v.}
\CommentTok{\# We can see that R did it "by column". That is, R first filled up the }
\CommentTok{\# first column of mat.v with the first two elements of vec.v, then moved}
\CommentTok{\# on to the second and third columns. }




\CommentTok{\# You can use the byrow argument to change this. This argument takes }
\CommentTok{\# one of two values, TRUE or FALSE (or T or F). That is, we write}
\CommentTok{\# matrix(data = v1, nrow = 2, byrow = TRUE)}
\CommentTok{\# Now, create an object called mat.w using the command above. }


\NormalTok{mat.w}\OtherTok{=}\FunctionTok{matrix}\NormalTok{(}\AttributeTok{data=}\NormalTok{v1,}\AttributeTok{nrow=}\DecValTok{2}\NormalTok{,}\AttributeTok{byrow=}\ConstantTok{TRUE}\NormalTok{)}

\FunctionTok{print}\NormalTok{(mat.v)}
\end{Highlighting}
\end{Shaded}

\begin{verbatim}
     [,1] [,2] [,3]
[1,]    1    5    9
[2,]    3    7   11
\end{verbatim}

\begin{Shaded}
\begin{Highlighting}[]
\FunctionTok{print}\NormalTok{(mat.w)}
\end{Highlighting}
\end{Shaded}

\begin{verbatim}
     [,1] [,2] [,3]
[1,]    1    3    5
[2,]    7    9   11
\end{verbatim}

\begin{Shaded}
\begin{Highlighting}[]
\CommentTok{\# Compare mat.v and mat.w. Do you see that R filled up the cells }
\CommentTok{\# "by row" to create the mat.w matrix ? }


\CommentTok{\# Many functions in R have arguments that take TRUE or FALSE like}
\CommentTok{\# the byrow argument we just used. In most cases, functions have a }
\CommentTok{\# default value. In the case of the matrix function, the default }
\CommentTok{\# value for the byrow argument is FALSE, meaning that, if you don\textquotesingle{}t}
\CommentTok{\# specify anything, R will automatically sets byrow = FALSE. }






\CommentTok{\# Find the number in the second row, second column of mat.w}


\NormalTok{mat.w[}\DecValTok{2}\NormalTok{,}\DecValTok{2}\NormalTok{]}
\end{Highlighting}
\end{Shaded}

\begin{verbatim}
[1] 9
\end{verbatim}

\begin{Shaded}
\begin{Highlighting}[]
\CommentTok{\# [1] 9}



\CommentTok{\# Find the number in the second row, second column of mat.v}


\NormalTok{mat.v[}\DecValTok{2}\NormalTok{,}\DecValTok{2}\NormalTok{]}
\end{Highlighting}
\end{Shaded}

\begin{verbatim}
[1] 7
\end{verbatim}

\begin{Shaded}
\begin{Highlighting}[]
\CommentTok{\# [1] 7}



\CommentTok{\# Finally, execute the entire contents of this R file by pressing}
\CommentTok{\# Ctrl + A and then pressing Ctrl + Enter.}
\CommentTok{\# Make sure that you don\textquotesingle{}t get any error message. If you get an }
\CommentTok{\# error message, it\textquotesingle{}s probably because you forgot to comment out }
\CommentTok{\# something. }


\CommentTok{\# End of file}
\end{Highlighting}
\end{Shaded}





\end{document}
